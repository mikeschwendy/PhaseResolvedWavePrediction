\documentclass[11pt]{amsart}
\usepackage{geometry}                % See geometry.pdf to learn the layout options. There are lots.
\geometry{letterpaper}                   % ... or a4paper or a5paper or ... 
%\geometry{landscape}                % Activate for for rotated page geometry
%\usepackage[parfill]{parskip}    % Activate to begin paragraphs with an empty line rather than an indent
\usepackage{graphicx}
\usepackage{amssymb}
\usepackage{epstopdf}
\DeclareGraphicsRule{.tif}{png}{.png}{`convert #1 `dirname #1`/`basename #1 .tif`.png}

\title{Phase-resolved Wave Prediction and Surface Reconstruction from SWIFT Buoy Array}
\author{Michael Schwendeman}
\date{30 April, 2017}                                           % Activate to display a given date or no date

\begin{document}
\maketitle
\section{Introduction}
- Project motivated by problem of wave energy conversion.  Much more efficient with advanced controls.  Most ideal: exact knowledge of the incident wave field.  Current practice: tune devices using only the long-time average wave statistics. \\
- Our question becomes: can we provide a better prediction of the incoming wave energy to a WEC then simply using the 30-minute average bulk wave parameters by making use of an array of buoys situated in front of the WEC?\\
- As a side question, which may be of more widespread applicability, can we construct an approximate sea surface reconstruction from the sparse buoy array?\\
\\
- This report documents the progress made so far on these questions.  It is broken into three sections:\\
1) Integration of SBG Ellipse into version 4 SWIFT buoys, and program for receiving data in real-time over RF ethernet bridge using python.\\
2) Development of phase-resolved algorithm using linear simulations from buoy directional spectra.\\
3) Evaluation of phase-resolving algorithm using real data obtained from a research cruise off of Southern California associated with the langmuir circulation DRI.\\
Finally, it closes with several recommendations and ideas for future work and directions.
%\subsection{}

\section{SBG and Ethernet Bridge Integration}
- Features of SWIFT v4: integration of Nortek Signature 5-beam ADCP.  Change in form factor.  Transition from microstrain to SBG inertial motion package.
- Features of SBG Ellipse: IMU/GPS/magnetometer, Real-time heave, Extended Kalman Filter data fusion.
- Features of ethernet bridge (frequency, range, etc.)

\section{Linear Simulations and Least Squares Prediction Algorithm}

\section{Evaluation from SWIFT data}

\section{Recommendations for Future Work}

\end{document}  